
\chapter{Théorie de la Fonctionnelle de la Densité.} % 

\label{Chapter2} 
%----------------------------------------------------------------------------------------
\section{Introduction}
Dans les années vingt de la dernière merde, le scientifique Schrödinger a exprimé le mouvement des électrons et des noyaux avec la phrase mathématique, cette expression est connue sous le nom de fonction d'onde et c'est la base de la physique quantique, mais cette équation a une solution précise sauf dans l'état de l'atome d'hydrogène, comme pour les atomes multi-électrons, les scientifiques ont été incapables de trouver des solutions à cette équation, ce qui a conduit à ils ont conduit à la découverte de la théorie dite de la fonctionnelle de la densité (DFT) ; ce qui est considéré comme l'une des méthodes les plus importantes utilisées en physique et en chimie en raison de son utilisation dans les calculs quantiques et cela est dû au fait qu'il peut être appliqué dans de nombreux domaines divers avec un coût élevé et une vitesse élevée, mais il fait toujours face à des difficultés de Afin d'utiliser les méthodes traditionnelles de résolution de l'équation de Schrödinger, c'est ce qui lui a fait connaître un développement remarquable dans les modifications des concepts de base de ces Récemment.
L'équation de Schrödinger est la base des études quantitatives de tout système cristallin quantique. Le système constitué de particules (ions + électrons) qui interagissent les unes avec les autres est décrit par l'équation de Schrödinger suivante:
\begin{equation}\label{2-1}
	H\psi = E\psi
\end{equation}
Ou:\\
$ H $:Hamiltonien.\\
$ \psi $ : La fonction d'onde du cristal\\

$ E $:L'énergie de l'état de base du cristal\\


L'hamiltonien total de cette phrase peut être considéré comme constitué de l'énergie cinétique de toutes les particules plus l'énergie d'interaction entre elles, et le cas échéant l'énergie d'interaction avec le milieu extérieur, pour écrire ce Hamiltonien en l'absence du champ extérieur sous la forme see l'équation suivante.
\begin{equation}\label{2-2}
	H_{int} = T_{e}+
\end{equation}
\begin{equation}\label{key}
	T_{e}= -\dfrac{\hbar^{2}}{2m}\sum_{i=1}^{N_{e}}\nabla_{i}^{2}
\end{equation}
L'équation de Schrödinger pour la structure cristalline contient un grand nombre d'inconnues (un nombre infini d'atomes et d'électrons), ce qui la rend impossible à résoudre car elle contient  variables, où  représente le nombre d'atomes, un exemple prend  d'un cristal qui contient  noyaux et le nombre Les variables deviennent environ , il est donc impossible de trouver une solution analytique ou numérique générale à cette équation, donc plusieurs approximations ont été faites pour simplifier cette équation.
\section{L’approximation de Born-Oppenheimer:}

jjhhh
llkkk