\chapter{Matériaux} % Main chapter title

\label{Chaputer1} % Change X to a consecutive number; for referencing this chapter elsewhere, use \ref{ChapterX}

%----------------------------------------------------------------------------------------
\section{Introduction:}
......
\section{Les terres rares}
Les ions de Les terres rares (TR) comprennent 14 métaux : des lanthanides du groupe  du tableau périodique, qui sont :(Cérium, Praséodyme, Néodyme, Prométhium, Samarium, Europium, Gadolinium, Terbium, Dysprosium, Holmium, Erbium, Thulium, Ytterbium, et Lutécium), comme il n'est pas considéré comme rare, contrairement à son nom, par exemple, le cérium a été trouvé dispersé dans la croûte terrestre comme certains autres métaux comme le cuivre\cite{a1}.

\begin{figure}[h!]
	\centering
	\includegraphics[width=0.9\linewidth, height=0.4\textheight]{Figures/Fig_I/1}
	\caption{Classification périodique des éléments de Mendeleïv.}
	\label{fig:1}
\end{figure}
\FloatBarrier

La série des lanthanides est constituée d'une série d'éléments successifs dans lesquels l'orbitale f est partiellement ou complètement remplie d'électrons et l'orbitale externe est vide, ce qui conduit à une physique intéressante et inhabituelle de celle-ci.

\begin{figure}[h!]
	\centering
	\includegraphics[width=0.9\linewidth, height=0.4\textheight]{Figures/Fig_I/2}
	\caption{Les orbitales f.}
	\label{fig:2}
\end{figure}
\FloatBarrier
Comme mentionné précédemment, ils ne sont pas particulièrement rares, du fait de leur nom dû à la difficulté de leur extraction et à leur découverte tardive, alors qu'il a fallu attendre les années quarante pour voir se développer des techniques d'extraction avancées. Il existe une légère différence entre les lanthanides uniquement avec le numéro atomique, ce qui rend difficile leur différenciation. Les terres rares se caractérisent par une conductivité électrique élevée et des points de fusion et d'ébullition élevés.

Les lanthanides sont caractérisés par l’existence d’électrons dans les orbitales 4f de faible extension spatiale. Elles sont écrantées de l’environnement extérieur par les orbitales de plus grande extension spatiale (, , , ), et ne participent pas à la liaison chimie.

La configuration électronique des terres rares peut s'écrire Par ($ Xe $) celle du xénon où n varie de 1($ Ce $) à 14 ($ Lu $).Quant au lanthane à 14 lutétium. Tous les lanthanides ont trois électrons dans Les orbitales $ 6s $ et $ 5d $ sont de tailles relativement grandes. Ces trois électrons ce sera donc très stressant en raison d'influences extérieures.

\begin{figure}[h!]
	\centering
	\includegraphics[width=0.9\linewidth, height=0.4\textheight]{Figures/Fig_I/3}
	\caption{Configuration électronique des lanthanides.}
	\label{fig:3}
\end{figure}
\FloatBarrier

\section{Nitrure de gallium ($ GaN $)}
Le nitrure de gallium est un matériau semi-conducteur binaire de la famille$  III-V $, les éléments qui le composent appartiennent à la 3e (Ga) et à la 5e (N) colonne de la classification périodique de Mendeleïev.

Les nitrures d’éléments $ III $ : $ GaN $,$  AlN $, $ InN $ et leurs alliages sont connus en tant que semi-conducteurs depuis les années 1960, époque à laquelle ils se sont largement répandus. On les retrouve par exemple dans l'électronique industrielle ou les dispositifs optoélectroniques. Ils ont été étudiés dans une large gamme pour leurs utilisations dans les détecteurs UV verts, bleus, violets et ultraviolets, ce qui élimine le besoin de filtres optiques coûteux\cite{a2}.

\subsection{Intérêt du nitrure de gallium}
Il existe plusieurs différences (caractéristiques) des matériaux nitrures par rapport aux autres, parmi ces avantages :

\begin{enumerate}
	\item Le GaN est un matériau d'une bande interdite directe (une large bande qui permet de travailler à haute tension et température), stabilité chimique élevée, ainsi que d'excellentes propriétés mécaniques, ses propriétés physiques intéressantes le rendent attrayant pour les électrons à haute température, haute fréquence et haute énergie, non seulement attrayant pour les émetteurs bleus. Sa large bande interdite permet également d'envisager son utilisation dans des photo-détecteurs UV insensibles au rayonnement visible du soleil \cite{a3}.
	\item Possibilité de réaliser des structures hétérogènes grâce à la discontinuité de la bande conductrice élevée ($ AlGaN $, $ InGaN $, $ AlGaInN $, $ BN $, ...).
	\item Les nitrures d’éléments III : $ GaN $, $ AlN $, $ InN $ et leurs alliages sont connus en tant que semi-conducteurs depuis les années 1960, époque à laquelle ils se sont largement répandus. On les retrouve par exemple dans l'électronique industrielle ou les dispositifs optoélectroniques. Ils ont été étudiés dans une large gamme pour leurs utilisations dans les détecteurs UV verts, bleus, violets et ultraviolets, ce qui élimine le besoin de filtres optiques coûteux \cite{a4}. 
	
\end{enumerate}

\section{Propriétés de GaN}
Le $ GaN $ est un semi-conducteur étonnant car il couvre tout le spectre visible ainsi que le rayonnement ultraviolet.Le nitrure de gallium cristallise sous deux formes différentes. Le poly type thermodynamiquement stable est la phase hexagonale (structure wurtzite(wz-$GaN $). Le poly type cubique (structure blende de zinc(zb-$ GaN $), thermodynamiquement métastable. GaN possède deux propriétés uniques : une grande vitesse de dérive des électrons saturés et une grande bande interdite directe. Nous allons introduire quelques propriétés de GaN\cite{a5}. 

\subsection{Propriétés structurelles}

Comme mentionné précédemment, le nitrure de gallium cristallise sous deux formes différentes : la forme hexagonale, appelée Wurtzite, ou la forme cubique, appelée Zinc Blende\cite{a5}.

La structure wurtzite indique dans la fig \ref{fig:4} , nommée d'après le minéral wurtzite. La structure wurtzite est constituée de deux réseaux hexagonaux, l'un avec les atomes de gallium, l'autre avec les atomes d'azote, interpénétrés et décalés suivant l'axe c de $ \frac{5}{8} $ 
de la maille élémentaire.pour lui Les paramètres réseau sont $a = \SI{3.189}{\angstrom}$ et $c = \SI{5.186}{\angstrom}$\cite{a6}.  sont spécifiés dans le tableau \ref{GaN-wz}.

  \begin{figure}[h!]
  	\centering
  	\includegraphics[width=0.9\linewidth, height=0.3\textheight]{Figures/Fig_I/4}
  	\caption{Structure cristalline du GaN dans sa phase (a) Zinc-Blende et sa phase (b) Wurtzite.}
  	\label{fig:4}
  \end{figure}


L’autre forme cristalline métastable du $ GaN $ est la forme cubique (dite Zinc Blende fig \ref{fig:4}. qui s'apparente à celle du diamant ($ C $, $ Ge $,$ Si $, ets...), qui se compose de deux réseaux cubiques à faces centrées (cubique a face centré)un occupé par les éléments III et l’autre occupé par les atomes d’azote, décalé d’un quart de la diagonale principale, c'est-à-dire de $ \frac{3}{4} $ a [111], où a, représente le paramètre de maille du matériau Cette structure Blende de Zinc, avec un paramètre de maille $a = \SI{4.511}{\angstrom}$ \cite{a8}.   sont spécifiés dans le tableau \ref{GaN-wz}.

\begin{table}
	\centering
	\begin{tabular}{ll}
		\hline\noalign{\smallskip}
		Structure cristalline & Huxagonale$ wurtzite $ \\
		\noalign{\smallskip}\hline\noalign{\smallskip}
		$ paramétre~ de~ maille $ & a= 3.189 \\
		~~~~~~ & c= 5.185 \\
		 $ groupe~ d'espace $ & P63mc  \\
		\noalign{\smallskip}\hline
	\end{tabular}
	\caption{paramètres cristallins de GaN type wurtzite \cite{a7} .}
	\label{GaN-wz} 
\end{table} 

\begin{table}
	\centering
	\begin{tabular}{ll}
		\hline\noalign{\smallskip}
		~~~GaN~~~ & \\
		\noalign{\smallskip}\hline\noalign{\smallskip}
		$ Structure cistailline $ & Cubique (zinc de blende) \\
	$ paramétre~ de~ maille $ & a= 4.511 \\
$ groupe d'espace $ & F43m \\
		\noalign{\smallskip}\hline
	\end{tabular}
	\caption{paramètres cristallins de GaN type de blende\cite {a8}.}
	\label{GaN-wz}
\end{table} 
\subsection{ Propriétés électriques}
Les propriétés thermiques d'un semi-conducteur sont spécialement déterminées par les
caractéristiques suivantes.

\begin{enumerate}
	\item Structures de bandes: comme la plupart des « nitrures d’éléments III », ($ InN $, $ GaN $, $ AlN $, mais pas $ BN $), la structure de bande des composés $ GaN $ présente un gap direct \cite{a15}. Pour un semi-conducteur à gap direct, le minimum ou les minima de la bande de conduction sont situés aux mêmes emplacements dans l’espace réciproque que le maximum ou les maxima de la bande de valence. Pour tous les semi-conducteurs des familles   IV ,III-V, II-VI, les 3 ou 4 bandes de valence présentent un maximum au centre de la zone de Brillouin (hexagonale pour le h-$ GaN $, cubique pour le c-$ GaN $), centre appelé $ « \tau » $, origine de l’espace réciproque et qui correspond à un état de vecteur d’onde nul pour les électrons. Pour les semi-conducteurs $ GaN $ et les autres semi-conducteurs à gap direct III-V et II-VI, la bande de conduction présente un seul minimum absolu situé aussi au centre de la zone de Brillouin.
	\item Masses effectives des porteurs:Dans le cas des semi-conducteurs à gap direct en coordinence tétraédrique, la masse effective d’un électron varie peu avec la direction de son vecteur d’onde. Un argument qui fournit une explication qualitative dans ce sens est le suivant. La structure 2H peut être considérée comme une succession de couches de symétrie cubiques tronquées, retournées un rang sur deux, et juxtaposées. La zone de Brillouin attendue dans ce cas est celle de la zone de Brillouin de la structure cubique, mais repliée selon l’axe  $ c $  dans une zone de hauteur moitié
	\item Transport électronique à faible champ électrique  mobilité des électrons et des trous: La mobilité électrique d’un porteur de charge représente la fonction de réponse (linéaire) de la vitesse moyenne des porteurs impliqués vis-à-vis du champ électrique moyen appliqué dans le cristal à une échelle typiquement supérieure au nanomètre. 
	Dans tous les matériaux GaN hétéro-épitaxies, à cause de la contrainte, et à plus forte raison dans le GaN hexagonal fondamentalement anisotrope, cette mobilité devrait être représentée par un tenseur de rang 2, symétrique en l’absence de champ électrique \cite{a13}.
 
\end{enumerate}


\subsection{Propriétés optiques }
Dans les semi-conducteurs, les propriétés optiques résultent des transitions
électroniques entre les niveaux de la bande de valence et de la bande de conduction.
Les transitions peuvent être directes ou indirectes, peuvent impliquer des interactions
entre les paires électron-trou et les niveaux énergétiques dus à des impuretés ou des défauts. Les propriétés optiques dépendent beaucoup de l’échantillon lui-même. Les
mesures optiques donnent des informations sur la structure et la composition.
Le nitrure de gallium est biréfringent. La connaissance de son indice de réfraction
est importante pour l’élaboration des structures des dispositifs d’optoélectronique. Il
est mesuré par ellipsométrie spectroscopique, réflectivité, transmission ou encore
luminescence dans le visible et l’infrarouge. Dans cette région, la partie imaginaire de
l’indice de réfraction est négligeable.
La valeur de sa partie réelle s’exprime de la manière suivante, en fonction de
l’énergie E de la source lumineuse :
\begin{equation}\label{key}
	n=1+\dfrac{1}{E^{2}_{0}-E^{2}}
	\end{equation}  
Où A et  $ E_{0} $ sont des constantes énergétiques 

Avec A=385 eV et  $ E_{0} $=9 eV \cite{a9}
 
Entre 900 et 2000nm, la dispersion est faible. En dessous de 800nm, elle augmente car l’énergie des phonons approche celle du gap. Il apparaît aussi que la dispersion est très sensible à la teneur en oxygène, elle diminue quand la concentration en oxygène augmente.
De manière générale, la valeur de l’indice de réfraction varie entre 2.1 et 2.5 à 1900 nm et 480 nm respectivement pour un film cristallin de $ GaN $ de bonne qualité \cite{a10}.
Le coefficient d’absorption a aussi fait l’objet de nombreuses études. La valeur ducarré est li néaire avec l’énergie des photons au-dessous de l’énergie du gap.
\begin{equation}\label{key}
	\alpha^{2}=\alpha^{2}_{0}(E-E_{g})
\end{equation} 
	avec $ \alpha_{0}=1.08 × 10^{5}cm^{-1} $

$ E_{g} $.Energie du gap 

Ceci confirme le fait que le gap du $ GaN $ est direct \cite{a11}]. Les mesures du coefficient
d’absorption de $ GaN $ de type wurtzite au gap excitonique donnent des valeurs de $ 3 × 10^{4}   ~~~~~ 1.5 × 10^{5}cm^{-1} $
à  en outre, il faut signaler que les mesures de réflectance en
UV/visible sont sensibles aux transitions excitoniques et inter bandes\cite{a10} 

\subsection{Les propriétés thermodynamiques et thermiques}
Les propriétés thermiques d'un semi-conducteur sont spécialement déterminées par les
caractéristiques suivantes.
\begin{enumerate}
	\item Dilatation thermiques: La connaissance du coefficient de dilatation thermique des semi-conducteurs est très importante pour la conception et l’élaboration des composants électroniques. Cependant, dans le cas de couches hétéro-épitaxies, si la dilatation thermique du substrat est différente de celle de la couche active, il pourra se créer des contraintes résiduelles induisant des fissures du substrat et de la couche épitaxie\cite{a12}.
	\item Capacité thermiques : La capacité thermique reflète la capacité d’une masse donnée d’un matériau à accumuler de l’énergie sous forme thermique quand sa température augmente. Dans le cas de GaN, la capacité thermique calculée à pression constante en fonction de la température est donnée par\cite{a13}. 
	\begin{equation}\label{key}
		c_{P}=38.1+8.96.10^{-3}.T
	\end{equation} 
	\item Conductivité thermiques: La conductivité thermique κ est la propriété cinétique qui est déterminée par les contributions des degrés de liberté électronique, rotationnel et vibrationnel. Dans les semi-conducteurs, à cause des faibles densités d’électrons de conduction et de trous, la contribution principale au transport de chaleur vient des phonons. Dans un cristal pur, la diffusion des phonons est le processus limitant. Dans un cristal réel, la conductivité thermique est déterminée par les défauts ponctuels ainsi que les joints de grain dans les céramiques. De toutes les propriétés thermiques, la conductivité est la plus affectée par les défauts de structure \cite{a14}. 
	
\end{enumerate}

\subsection{Propriétés mécaniques}
Lors de l’épitaxie d’une hétéro structure, comme un puits quantique de $ GaN $/$ AlGaN $,on cherche à faire des couches cohérentes c'est-à-dire avec continuité du paramètre de maille pour éviter la formation de dislocations qui dégradent les propriétés optiques des échantillons. Dans ce cas il faut que le désaccord de paramètres de maille soit accommodé par une déformation du réseau cristallin. La couche déposée emmagasine alors une énergie élastique  jusqu’à une épaisseur limite au-delà de laquelle il est plus avantageux énergétiquement pour elle de relaxer cette contrainte en formant des dislocations (ou dans certaines conditions des boîtes quantiques). Cette épaisseur est appelée épaisseur critique, et elle est d’autant plus fine que le désaccord de paramètre de maille est grand. La différence relative de paramètre de maille entre $ GaN $ et $ AlN $ est de 2.4 (variant selon les orientations). Par ailleurs, le substrat peut lui aussi induire une contrainte dans la couche épitaxiée quand il s’agit d’un matériau différent de celui de la couche tampon \cite{a17}. Les propriétés des semi-conducteurs dépendent dans une large mesure de leur état de contrainte et des déformations locales ou globales du réseau cristallin qui y sont liées. En effet, toute déformation entraîne une modification des positions relatives des atomes les uns par rapport aux autres et donc du recouvrement des orbitales atomiques. Il s’ensuit une modification du diagramme de bandes et en particulier de la largeur de la bandeinterdite (gap). 

